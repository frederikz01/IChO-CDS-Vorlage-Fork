\def\solution{2} % 0: Klausurvariante, 1: Sch\"ulerversion, 2: Betreuerversion

%Ab hier Finger weg, hier sollte bis zum finalen Satz nichts angefasst werden muessen!
\def\klasse{10}
\def\bundesland{Mecklenburg-Vorpommern}
\def\cyear{2025}
\def\round{2} % 1, 2, 3, 4
\def\exam{2} % 1, 2; Nur fuer 3. & 4. Runde
\def\fullname{57. IChO 2025 in den Vereinigten Arabischen Emiraten} %z.B. 56. IChO 2024 in Saudi-Arabien (f\"uer Deckblatt)
\def\code{1} %0: Linie; 1: aus Datei importieren

\def\paperoption{plainpaper_Cds}
\documentclass[\paperoption]{ichobooklet}
\begin{document}%\serienbriefstart %Auslagerung von \loop funktioniert nicht, die auskommentierten Befehle sind Quelltextleichen
\newread\quelle
 \openin\quelle=\opt{c1}{Codes.csv}\opt{c0}{./Hilfsdateien/Dummycode.csv}
 \loop
 \read\quelle to \zeile
  \ifeof\quelle
   \global\morefalse
  \else
   \expandafter\chopline\zeile\\

\setcounter{page}{1}\setcounter{section}{0}
\subfile{./Hilfsdateien/deckblatt}
\subfile{./Hilfsdateien/formelsammlung}
\subfile{./Hilfsdateien/periodensystem}
\subfile{./Hilfsdateien/NMR}

%%%%%%%%%%%%%%%%%%
% 1. Aufgabe
\subfile{2025-3-11}

% 2. Aufgabe
\subfile{2025-3-12}

% 3. Aufgabe
\subfile{2025-3-13}

% 4. Aufgabe
\subfile{2025-3-14}

\subfile{To-do}

%\serienbriefende
  \fi
 \ifmore\repeat
 \closein\quelle
\end{document}